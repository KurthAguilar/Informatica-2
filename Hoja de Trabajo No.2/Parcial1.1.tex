\documentclass[10pt,a4paper]{article}
\usepackage[utf8]{inputenc}
\usepackage[spanish]{babel}
\usepackage{amsmath}
\usepackage{amsfonts}
\usepackage{amssymb}
\topmargin=-0.45in
\evensidemargin=0in
\oddsidemargin=0in
\textwidth=6.5in
\textheight=9.0in
\headsep=0.25in
\linespread{1.1} 
\renewcommand{\baselinestretch}{1.5}
\begin{document}
\title{Parcial 1.1}
\author{Kurth Aguilar - 20181242 }
\date{14 de agosto de 2018}
\maketitle
\section*{Solucion Problema 1}
\begin{itemize}
\item \textbf{Nodos: } \ $ \left\lbrace 1,2,3,4,5,6,7\right\rbrace $
\item \textbf{Vertices: }  
 $$
    \left\langle \left\{
        \begin{bmatrix}
            \langle 1,2 \rangle & \langle 1,3 \rangle & \langle 1,4 \rangle \\
            \langle 1,5 \rangle & \langle 1,6 \rangle & \langle 2,3 \rangle \\
            \langle 2,4 \rangle & \langle 2,5 \rangle & \langle 2,6 \rangle \\
            \langle 3,4 \rangle & \langle 3,5 \rangle & \langle 3,6 \rangle \\
            \langle 3,7 \rangle & \langle 4,7 \rangle & \langle 5,6 \rangle \\
            \langle 5,7 \rangle & \langle 6,7 \rangle &   \\
        \end{bmatrix}
    \right\} \right\rangle
$$
\end{itemize}
\section*{Solucion Problema 2}
\begin{center}
$ \sum_{i=1}^{n}{i}=\frac{n(n+1)}{2} $
\end{center}
\begin{itemize}
\item \textbf{Caso Base: } \ $ n = 1 $ 
\begin{center}
\
\\ $ 1 = \frac{1(1+1)}{2} $
\
\\ $ 1 = \frac{1(2)}{2}$
\
\\ $ 1 = \frac{2}{2}$
\
\\ $ 1 = 1 $
\end{center}
\item \textbf{Caso Inductivo: } $ \forall \ n$
\begin{itemize}
\item \textbf{Hipotesis Inductiva: } \ Suponiendo que el Axioma $ p5 $ de Peano se cumple, en el cual dice que una propiedad $ p(n) $ se cumple en los sucesores si el predecesor cumple con la propiedad. Entonces si suponemos que $ p(n) $ es verdadera nuestra hipotesis inductiva seria: 
\begin{center}
$ \sum_{i=1}^{n}{i}=\frac{n(n+1)}{2} $
\end{center}
\item \textbf{Sucesor: } \ $ n+1 $
\item \textbf{Demostracion: }
\begin{center}
$ \sum_{i=1}^{n}{i}= \frac{n+1(n+1+1)}{2} $
\
\\$					= \frac{n+1(n+2)}{2} $
\
\\$					= \frac{n+1 [(n+1)+1]}{2} $
\
\\\textbf{*Nota:} \ Con esta induccion o demostracion, se puede cumplir la hipotesis inductiva $ \forall n $, en la cual la propiedad $ p(n) $ se cumple. 
\end{center}
\end{itemize}
\end{itemize}

\section*{Solucion Problema 3}
\begin{center}
$ \sum(n)=1+2+3+4+\ \ldots\ +n $
\
\\ $ \sum(n)   \left\{
                        \begin{array}{ll}
                                0  & \mbox{si } n = 0 \\
                                s(i\oplus b) & \mbox{si } a = s(i)
                        \end{array}
                \right.
$
\end{center}
\section*{Solucion Problema 4}
\begin{center}
$ a\oplus b = b\oplus a $
\end{center}
\begin{itemize}
\item \textbf{Caso Base:} 
\begin{itemize}
\item \ $ a=0$
\begin{center}
 $ 0\oplus b = b\oplus 0$
 \
 \\ $ b=b$
\end{center}
\item \ $b=0$
\begin{center}
$ a\oplus 0 = 0\oplus a $
\
\\ $ a=a$
\end{center}

\end{itemize}
\item \textbf{Caso Inductivo: } \ $ \forall \ a \bigwedge b $
\begin{itemize}
\item \textbf{Hipotesis Inductiva: } 
\begin{center}
$ s(a)\oplus b = b\oplus s(a) $
\end{center}
\end{itemize}

\end{itemize}


\end{document}
\documentclass[10pt,a4paper]{article}
\usepackage[utf8]{inputenc}
\usepackage{amsmath}
\usepackage{amsfonts}
\usepackage{amssymb}
\topmargin=-0.45in
\evensidemargin=0in
\oddsidemargin=0in
\textwidth=6.5in
\textheight=9.0in
\headsep=0.25in
\linespread{1.1} 
\renewcommand{\baselinestretch}{1.5}
\begin{document}
\title{Hoja de Trabajo No.2}
\author{Kurth Michael Aguilar Ecobar - 20181242}
\maketitle
\section*{Ejercicio No.1}
\begin{large}
\textbf{Instrucciones:} \ {Demostrar utilizando induccion}  
\end{large}
\[  
	\forall\ n.\ n^3\geq n^2
\]
donde $n\in\mathbb{N}$

\
\\\begin{large}
\textbf{Solucion:}
\end{large}

\begin{itemize}
\item \textbf{Caso Base:} \ n = 0
 \[ \ 0^{3}\geq 0^{2}\ = 0\geq 0 \]

\item \textbf{Caso Inductivo $n\in\mathbb{N}$}

 Hipotesis inductiva: \[
        \ n^3\geq n^2 \ = n*(n^2)\geq (n^2)
\]
 Sucesor: S(n) = (n+1) 
 
\
\\\textbf{Demostracion:}
\
\\ $ (n+1)(n+1)^2\geq (n+1)^2 $ 
\
\\ $ (n+1)\geq \frac{(n+1)^2}{(n+1)^2}$
\
\\ $ (n+1)\geq 1$
\
\\ $ n+1\geq 1 $
\
\\ $ n\geq 1-1 $
\
\\ $ n\geq 0 $ 

\
\\\textbf{Conclusion:}
\ En la demostracion sobre el caso inductivo se puede apreciar que no se demuestra totalmente la hipotesis inductiva. Eso sucede porque hay casos en cuales la demostracion no llega a la hipotesis inductiva, sin embargo se pudo demostrar que $ \forall\ n$ siempre se cumplira la propiedad establecida, porque siempre sera mayor o igual que 0. 



\end{itemize}

\section*{Ejercicio No.2}

\begin{large}
\textbf{Instrucciones:} \ {Demostrar utilizando induccion la desigualdad de Bernoulli.}
\end{large}

\begin{center}
$ \forall\ n.(1+x)^n \geq nx $
\end{center}
\
donde $n\in \mathbb{N}$, $x\in \mathbb{Q}$ y $x\geq -1$

\
\\\begin{large}
\textbf{Solucion: }
\end{large}

\begin{itemize}
\item \textbf{Casos Base:} \ $ n=0 $ \ y \ $ x=0 $
\begin{center}
\textbf{n = 0}
\
\\$ (1+x)^0 \geq 0*x $
\
\\$ 1 \geq 0 $

\
\\\textbf{x = 0}
\
\\ $ (1+0)^n \geq n*0 $
\
\\ $ 1^n \geq 0 $
\
\\\textbf{*Nota:} \ Recordar que $ n=0 $ con lo cual $ 1^n $ es igual a 1.
\end{center}

\item \textbf{Caso Inductivo:} \ $n\in \mathbb{N}$, $x\in \mathbb{Q}$ y $x\geq -1$

\ Hipotesis Inductiva:

\ Sucesor: 
\end{itemize}






\end{document}
\documentclass[10pt,a4paper]{article}
\usepackage[utf8]{inputenc}
\usepackage[spanish]{babel}
\usepackage{amsmath}
\usepackage{amsfonts}
\usepackage{amssymb}
\topmargin=-0.45in
\evensidemargin=0in
\oddsidemargin=0in
\textwidth=6.5in
\textheight=9.0in
\headsep=0.25in
\linespread{1.1} 
\renewcommand{\baselinestretch}{1.5}
\begin{document}
\title{Hoja de Trabajo No.3}
\author{Kurth Michael Aguilar Ecobar - 20181242}
\date{ Fecha de entrega: 16 de agosto de 2018 }
\maketitle


\section*{Ejercicio No.1}

\begin{flushleft}
\textbf{Instrucciones:} \ Utilizando la definicion de suma ($\oplus$) para los numeros naturales unarios, llevar
a cabo la suma entre tres [$s(s(s(0)))$] y cuatro [$s(s(s(s(0))))$]. Debe elaborar todos
los pasos de forma explicita. Como referencia, se presenta nuevamente la definici\'on de
suma para numeros naturales unarios:
\end{flushleft}

\begin{center}
$ n\oplus m := \left\{ 
\begin{array}{l l}
 m & \mbox{si } n=o \\ 
 n & \mbox{si } m=o \\ 
 s(i\oplus m) & \mbox{si } n=s(i) \\ 
 \end{array}
 \right.$
\end{center}

\
\\\begin{large}
\textbf{Solucion:}
\end{large}
\begin{itemize}
\item En este caso se plantea la expresión de $ n\oplus m $ donde $ m $ debe ser sucesor de cualquier numero: $ s(i) $. Donde nuestro $ s(i) $ equivale a   \ $ s(s(s(s(0))))  $. 
\item $n$ equivale a tres $ \left[ s(s(s(0))) \right]  $. 
\item \textbf{Explicacion de la suma entre $ (n\oplus s(i)) $}
\begin{center}
$ n\oplus m $
\
\\$n\oplus s(i)$
\
\\$( s(s(s(0)))) \oplus s(s(s(s(0)))) \ = \ s[ s(s(s(0)))) \oplus s(s(s(0)))]$
\
\\$ s(s(s(s(s(0)))) \oplus s(s(0))))$
\
\\$ s(s(s(s(s(s(0)))) \oplus s(0))))$
\
\\$ s(s(s(s(s(s(s(0)))) \oplus 0) \Rightarrow $ \ Esta forma demuestra el caso en el cual $ n\oplus 0 \ = \ n$
\
\\ $[s(s(s(s(s(s(s(0)))))))] \ =  $ \ A la suma buscada de  $ cuatro\oplus tres$ 
\end{center}
\end{itemize}

\section*{Ejercicio No.2}
\begin{flushleft}
\textbf{Instrucciones: } \ Definir inductivamente una funci\'on para multiplicar ($\otimes$) numeros naturales unarios.
\end{flushleft}
\begin{center}
$ n\otimes m := \left\{
\begin{array}{l l}
0 & \mbox{si } n=o \ \vee \ m=0 \\ 
 n\otimes s(m) & \mbox{si } n\oplus n\otimes m\\ 
 \end{array}
 \right.$
 
 \
 \\\textbf{*Nota: } \ En este caso en la definicion de la multiplicacion con numeros unarios, se definieron los casos posibles que incluyen un caso base $ = 0 $, para demostrar la solucion en la presentacion de dicho caso. Al igual que en esta definicion se presenta la definicion para el sucesor de un numero en este caso $ s(m) $, para de esa manera generalizar los casos posibles de cualquier numero unario. Como se aprecia, se utilizo el operado de suma $ \oplus $, que solo significa  que tenemos la necesidad de eliminar mas de un operador.  La expresion $n\oplus n\otimes m$ es una suma unaria que tiene como caracteristica un producto en su segunda expresion. Para operarlo se puede usar las operaciones aritmeticas ya conocidas. 
 \end{center}

\section*{Ejercicio No.3}
\begin{flushleft}
\textbf{Instrucciones: } \ Verifique que su definici\'on de multiplicaci\'on es correcta multiplicando los siguientes valores:
\end{flushleft}
\begin{itemize}
\item{$s(s(s(0)))\otimes 0$}
\begin{itemize}
\item \textbf{Verificacion: }
\begin{center}
\textbf{*Nota: } \ En este caso la presentacion de la operacion con la expresion $n\oplus n\otimes m$, es imposible, porque de acuerdo a los Axiomas de Peano $ 0 $ no tiene predecesor. Sin embargo, en la definicion de la multiplicacion se definio que en el caso de que $ m \vee n = 0$ el resultado daria $ 0 $. 
\end{center}
\end{itemize}

\item{$s(s(s(0)))\otimes s(0)$}
\begin{itemize}
\item \textbf{Verificacion: }
\begin{center}
$s(s(s(0)))\otimes s(0) = s(s(s(0)))\oplus s(s(s(0)))\otimes 0$
\
\\ $\dots = s(s(s(0)))\oplus 0$
\
\\ $\dots = s(s(s(0)))$
\
\\$s(s(s(0))) = s(s(s(0)))$
\end{center}
\end{itemize}
\item{$s(s(s(0)))\otimes s(s(0))$}
\begin{itemize}
\item \textbf{Verificacion: }
\begin{center}
$ s(s(s(0)))\otimes s(s(0)) = s(s(s(0)))\oplus s(s(s(0)))\otimes s(0)$
\
\\$\dots = s(s(s(0)))\oplus s(s(s(0)))$
\
\\ $\dots = s(s(s(s(s(s(0))))) $
\
\\ $s(s(s(s(s(s(0)))))  = s(s(s(s(s(s(0))))) $
\end{center}
\end{itemize}
\end{itemize}
\section*{Ejercicio No.4}
\begin{flushleft}
\textbf{Instrucciones: } \ Demostrar utilizando induccion.
\end{flushleft}
\begin{itemize}
        \item{$a\oplus s(s(0))=s(s(a))$}
        \begin{itemize}
        \item \textbf{Demostracion: }
        \begin{itemize}
        \item \textbf{Caso Base: } \ $ a = 0 $
        \begin{center}
        $0\oplus s(s(0))=s(s(0))$
        \
        \\$ s(s(0))=s(s(0))$
        
        \end{center}
        \item \textbf{Caso Inductivo: } \ $ \forall\ a$
        \begin{itemize}
        \item \textbf{Hipotesis Inductiva: } \ $a\oplus s(s(0))=s(s(a))$
        \item \textbf{Sucesor: } \ $ s(a) \ \vee \ \ a\oplus 1 $
        \begin{center}
        $s(a)\oplus s(s(0)) = s(s(s(a))) $
        \
        \\ $s(a)\ominus s(0)\oplus s(s(0)) = s(s(s(a)))\ominus s(0) $
        \
        \\$ a\o\triangleleft\triangleleft\triangleleftplus s(s(0)) = s(s(s(a)))\ominus s(0) $
        \
        \\ $a\oplus s(s(0))=s(s(a))$
        \end{center}
        \end{itemize}
        \end{itemize}
        \end{itemize}
        \item{$a \otimes b = b \otimes a$}
        \begin{itemize}
        \item \textbf{Demostracion: }
        \begin{itemize}
        \item \textbf{Caso Base: } \ $ a = 0$
        \begin{center}
        $0\otimes b = b\otimes 0$
        \
        \\
        $ 0 = 0$
       
        \end{center}
         \item \textbf{Caso Inductivo: } \ $ \forall\ a $
         \begin{itemize}
        \item \textbf{Hipotesis Inductiva: } \ $ a\otimes b = b\otimes a$
        \item \textbf{Sucesor: }$ a+1$
        \begin{center}
        $  (a\oplus 1)\otimes b = b\otimes (a\oplus1)$
        \
        \\ $ ab \oplus b = ba \oplus b$
        \
        \\ $ ab \oplus b \ominus b= ba \oplus b\ominus b $
        \
        \\ $ ab = ba $
        \
        \\ $ a\otimes b = b\otimes a $
        \end{center}
         \end{itemize}
        \end{itemize}
        \end{itemize}
        \item{$a \otimes (b \otimes c)=(a\otimes b)\otimes c$} 
        \begin{itemize}
        \item \textbf{Demostracion: }
		\begin{itemize}
		\item \textbf{Caso Base:} \ $ c = 0 $
		\begin{center}
		$ a \otimes (b \otimes 0)=(a\otimes b)\otimes 0$
		\
		\\ $ a\otimes (0) =(a\otimes b)\otimes 0 $
		\
		\\ $ 0 = 0 $
	
		\end{center}
		\item \textbf{Caso Inductivo: } \ $\forall\ c $
		\begin{itemize}
		\item \textbf{Hipotesis Inductiva: } \ $ a \otimes (b \otimes c)=(a\otimes b)\otimes c$
		\item \textbf{Sucesor: } \ $ c+1$
		\begin{center}
		$a \otimes (b \otimes (c\oplus 1))=(a\otimes b)\otimes (c\oplus 1)$
		\
		\\$ a\otimes(bc\oplus b) = abc\oplus ab$
		\
		\\$ abc\oplus ab = abc\oplus ab $
		\
		\\ $ abc\oplus ab\ominus ab = abc\oplus ab\ominus ab $
		\
		\\$ abc = abc $
		\
		\\ $ a\otimes b\otimes c= a\otimes b\otimes c $
		\
		\\ $ a \otimes (b \otimes c)=(a\otimes b)\otimes c$
		\end{center}
		\end{itemize}
		\end{itemize}
\end{itemize}               
       
       
        \item{$(a\oplus b)\otimes c = (a\otimes c) \oplus (b \otimes c)$}
        \begin{itemize}
        \item \textbf{Demostracion:}
        \begin{itemize}
        \item \textbf{Caso Base: } \ $ c=0 $
        \begin{center}
        $(a\oplus b)\otimes 0 = (a\otimes 0) \oplus (b \otimes 0)$
        \
        \\ $ (a0\oplus b0) = (0)\oplus (0) $
        \
        \\ $ (0\oplus 0) = 0 $
        \
        \\ $ 0 = 0 $
        \end{center}
        \item \textbf{Caso Inductivo: } \ $ \forall\ c$ 
        \begin{itemize}
        \item \textbf{Hipotesis Inductiva: } \ $(a\oplus b)\otimes c = (a\otimes c) \oplus (b \otimes c)$
        \item \textbf{Sucesor:} \ $ c+1$
        \begin{center}
        $(a\oplus b)\otimes (c\oplus 1) = (a\otimes (c\oplus 1)) \oplus (b \otimes (c\oplus 1))$
        \
        \\ $ ac\oplus bc\oplus a\oplus b = (ac\oplus a)\oplus (bc\oplus b)$
        \
        \\ $ ac\oplus bc\oplus a\oplus b = ac\oplus bc\oplus a\oplus b $
        \
       	\\ $ ac\oplus bc\oplus a\oplus b\ominus a\ominus b = ac\oplus bc\oplus a\oplus b\ominus a\ominus b $
       	\
       	\\ $ ac\oplus bc\ = ac\oplus bc $
       	\
       	\\$ (a\oplus b)\otimes c = ac\oplus bc $
       	\
       	\\ $(a\oplus b)\otimes c = (a\otimes c) \oplus (b \otimes c)$
        \end{center}
        \end{itemize}
        \end{itemize}
        \end{itemize}
\end{itemize}



\end{document}
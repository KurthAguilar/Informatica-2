\documentclass[10pt,a4paper]{article}
\usepackage[utf8]{inputenc}
\usepackage[spanish]{babel}
\usepackage{amsmath}
\usepackage{amsfonts}
\usepackage{amssymb}
\topmargin=-0.45in
\evensidemargin=0in
\oddsidemargin=0in
\textwidth=6.5in
\textheight=9.0in
\headsep=0.25in
\linespread{1.1} 
\renewcommand{\baselinestretch}{1.5}
\begin{document}
\title{Hoja de Trabajo No.3}
\author{Kurth Michael Aguilar Ecobar - 20181242}
\date{ Fecha de entrega: 16 de agosto de 2018 }
\maketitle


\section*{Ejercicio No.1}

\begin{flushleft}
\textbf{Instrucciones:} \ Utilizando la definicion de suma ($\oplus$) para los numeros naturales unarios, llevar
a cabo la suma entre tres [$s(s(s(0)))$] y cuatro [$s(s(s(s(0))))$]. Debe elaborar todos
los pasos de forma explicita. Como referencia, se presenta nuevamente la definici\'on de
suma para numeros naturales unarios:
\end{flushleft}

\begin{center}
$ n\oplus m := \left\{ 
\begin{array}{l l}
 m & \mbox{si } n=o \\ 
 n & \mbox{si } m=o \\ 
 s(i\oplus m) & \mbox{si } n=s(i) \\ 
 \end{array}
 \right.$
\end{center}

\
\\\begin{large}
\textbf{Solucion:}
\end{large}
\begin{itemize}
\item En este caso se plantea la expresión de $ n\oplus m $ donde $ m $ debe ser sucesor de cualquier numero: $ s(i) $. Donde nuestro $ s(i) $ equivale a   \ $ s(s(s(s(0))))  $. 
\item $n$ equivale a tres $ \left[ s(s(s(0))) \right]  $. 
\item \textbf{Explicacion de la suma entre $ (n\oplus s(i)) $}
\begin{center}
$ n\oplus m $
\
\\$n\oplus s(i)$
\
\\$( s(s(s(0)))) \oplus s(s(s(s(0)))) \ = \ s[ s(s(s(0)))) \oplus s(s(s(0)))]$
\
\\$ s(s(s(s(s(0)))) \oplus s(s(0))))$
\
\\$ s(s(s(s(s(s(0)))) \oplus s(0))))$
\
\\$ s(s(s(s(s(s(s(0)))) \oplus 0) \Rightarrow $ \ Esta forma demuestra el caso en el cual $ n\oplus 0 \ = \ n$
\
\\ $[s(s(s(s(s(s(s(0)))))))] \ =  $ \ A la suma buscada de  $ cuatro\oplus tres$ 
\end{center}
\end{itemize}

\section*{Ejercicio No.2}
\section*{Ejercicio No.3}
\section*{Ejercicio No.4}


\end{document}